\documentclass[11pt,class=report,crop=false]{standalone}
\usepackage[screen]{../python}


\begin{document}


%====================================================================
\chapitre{Le compte est bon}
%====================================================================



\objectifs{Qui n'a jamais rêvé d'épater sa grand-mère en gagnant à tous les coups au jeu \og{}Des chiffres et des lettres\fg{} ? Une partie du jeu est \og{}Le compte est bon\fg{} dans lequel il faut atteindre un total à partir de chiffres donnés et des quatre opérations élémentaires. L'autre partie du jeu est \og{}Le mot le plus long\fg{}, cette fois il faut trouver le plus long mot français à partir d'un tirage de lettres.  Pour ces deux jeux les ordinateurs sont plus rapides que les humains, il ne te reste plus qu'à écrire les programmes !}


\bigskip
%Les programmations des deux jeux sont indépendantes. On en profite pour résoudre des problèmes proches. 

La résolution complète du \og{}Compte est bon\fg{} utilise un algorithme récursif.


Quatre activités sont proposées :
\begin{enumerate}
  \item on commence par générer un tirage de nombres ;
  \item on continue par un algorithme très simple qui teste beaucoup de solutions mais ne fonctionne pas pour certains cas ;
  \item on continue avec un problème plus simple : atteindre une somme fixée avec des nombres donnés ;
  \item enfin on mixe nos deux activités pour résoudre complètement notre jeu.
\end{enumerate}


%%%%%%%%%%%%%%%%%%%%%%%%%%%%%%%%%%%%%%%%%%%%%%%%%%%%%%%%%%%%%%%%
%%%%%%%%%%%%%%%%%%%%%%%%%%%%%%%%%%%%%%%%%%%%%%%%%%%%%%%%%%%%%%%%

\begin{cours}[Le compte est bon]
On se donne une liste d'entiers et un total à atteindre. Il faut réaliser ce total à l'aide des opérations \og{}$+$\fg{}, \og{}$-$\fg{}, \og{}$\times$\fg{} et \og{}$/$\fg{}.
Exemple :
$$[1, 3, 6, 8, 75, 100]  \qquad T=524$$
Une solution : $6-1=5$, puis $5 \times 100 = 500$, puis $3\times 8 = 24$, $500+24 = 524$ !

Voici les règles du jeu :
\begin{itemize}
  \item le total à atteindre est un entier tiré au hasard entre $100$ et $999$,
  
  \item on dispose de $6$ plaques tirées au hasard sur lesquelles sont inscrits des entiers,
  
  \item les plaques sont tirées parmi $28$ plaques (elles sont en double) :
  $$1,1,2,2,3,3,4,4,5,5,6,6,7,7,8,8,9,9,10,10,25,25,50,50,75,75,100,100$$
   
  \item chacune des $6$ plaques ne peut être utilisée qu'une seule fois (mais on n'est pas obligé d'utiliser toutes les plaques),
  
  \item les seules opérations autorisées sont \og{}$+$\fg{}, \og{}$-$\fg{}, \og{}$\times$\fg{} et \og{}$/$\fg{},
  
  \item les calculs ne se font qu'avec des entiers positifs (les soustractions doivent donner un résultat positif, les divisions doivent avoir un reste nul). 
\end{itemize}

Il n'est pas toujours possible de trouver le bon résultat. Dans ce cas, on cherche à s'approcher le plus possible du total.

\end{cours}



%%%%%%%%%%%%%%%%%%%%%%%%%%%%%%%%%%%%%%%%%%%%%%%%%%%%%%%%%%%%%%%%
% Activité 1 - Tirage
%%%%%%%%%%%%%%%%%%%%%%%%%%%%%%%%%%%%%%%%%%%%%%%%%%%%%%%%%%%%%%%%

\begin{activite}[Tirage]
   
\objectifs{Objectifs : programmer le tirage du total et des plaques.}
 
\begin{enumerate}
  \item Programme une fonction \ci{tirage_total()} qui renvoie un entier au hasard compris entre $100$ et $999$.
  
  \emph{Indication.} Utilise la fonction \ci{randint(a,b)} du module \ci{random}.
  
  \item Programme une fonction \ci{tirage_chiffres()} qui renvoie une liste de $6$ entiers tirés au hasard 
  parmi la liste des plaques possibles. 
  
  \emph{Exemple de tirage :} $[3,5,7,7,25,100]$.
     
  \emph{Indication.} Il s'agit d'un tirage sans remise : une plaque tirée est ensuite écartée pour le reste du tirage.
  Comme l'ordre n'a pas d'importance pour le jeu, on peut renvoyer une liste ordonnée.
  
  
  \item Programme une fonction \ci{operation(a,b,op)} où $a$ et $b$ sont deux nombres et \ci{op} est un caractère parmi \ci{'+'}, \ci{'-'}, \ci{'*'}, \ci{'/'} et qui renvoie le calcul demandé parmi $a+b$, $a-b$, $a \times b$, $a/b$.
  
  \emph{Exemple.} \ci{operation(7,5,'-')} renvoie $7-5=2$.
  
\end{enumerate}

\end{activite}



%%%%%%%%%%%%%%%%%%%%%%%%%%%%%%%%%%%%%%%%%%%%%%%%%%%%%%%%%%%%%%%%
% Activité 2 - Recherche basique
%%%%%%%%%%%%%%%%%%%%%%%%%%%%%%%%%%%%%%%%%%%%%%%%%%%%%%%%%%%%%%%%

\begin{activite}[Recherche basique]
   
\objectifs{Objectifs : programmer simplement une recherche brutale mais pas complète.}
 
 
\textbf{Solutions séquentielles.}
 
Dans un premier temps on va chercher seulement les solutions séquentielles, dans lesquelles les calculs s'enchaînent de gauche à droite.
Par exemple avec : $[2,3,4,5]$ 
\begin{itemize} 
  \item on peut obtenir séquentiellement $29$ par le calcul
  $$\big((2+3)\times 5\big) + 4 = 29$$
  c'est-à-dire que l'on fait les calculs un par un  en repartant à chaque fois du résultat précédent : $2+3=5$, puis $5 \times 5 = 25$, puis $25+4=29$ ;
  \item par contre on ne peut obtenir séquentiellement le total de $49$,  
  même si on pourrait l'obtenir par les règles normales : 
  d'une part $2 + 5 = 7$, d'autre part $3+4 = 7$ et on mixe ces deux sous-résultats $7\times 7 = 49$.
\end{itemize}
  

\bigskip

\textbf{Avec deux chiffres.}

Voici comment calculer toutes les opérations $c_1 + c_2$ et $c_1 \times c_2$ à partir d'une liste de chiffres et afficher si le total souhaité (ici $18$) est atteint.
 
\begin{lstlisting}
chiffres = [2,3,5,6,8,10]           # Plaques disponibles
total = 18                          # Objectif

chiffres1 = list(chiffres)          # Copie
for c1 in chiffres1:                # Premier chiffre
    chiffres2 = list(chiffres1)     # Copie
    chiffres2.remove(c1)            # Retirer la plaque déjà choisie

    for op in ['+','*']:            # Opérations

        for c2 in chiffres2:        # Second chiffre
        
            calcul = operation(c1,c2,op)   # Résultat du calcul            

            if calcul == total:            # Total atteint ?
                print("Trouvé !",c1,op,c2,'=',calcul)
\end{lstlisting}

\begin{enumerate}
  \item Analyse ces ligne de code et transforme-le en une fonction \ci{deux_plaques(total,chiffres)} qui gère cette fois les quatre types d'opérations. Est-ce que le total de $18$ peut être atteint avec les chiffres disponibles $[2,3,5,6,8,10]$ ? Si oui, de combien de façons différentes ?
  
  \item Modifie ton programme pour rechercher tous les calculs possibles avec trois plaques, c'est-à-dire $(c_1 \square c_2) \circ c_3$, où $\square$ et $\circ$ sont des opérations parmi les quatre opérations autorisées.
  
  \item Persévère avec une fonction \ci{recherche_basique(total,chiffres)} qui teste si on peut obtenir le total demandé à l'aide des $6$ plaques données par la liste \ci{chiffres}.
  
\emph{Indications.} Il faut imbriquer beaucoup de boucles ! On ne s'occupe pas encore des règles strictes du \og{}Compte est bon\fg{}, on s'autorise des soustractions négatives et des divisions non entières.
  
  Exemples :
  \begin{itemize}
    \item Laisse la machine obtenir $809$ à partir de $[2, 3, 6, 8, 75, 100]$.
    \item Atteins $779$ avec $[5, 6, 7, 8, 25, 50]$.
    \item Peux-tu trouver $773$ avec $[2, 4, 6, 8, 10, 50]$ ? Et $769$ avec ces mêmes chiffres ?
    
  \end{itemize}
  
  \item On se donne un tirage de $6$ plaques. Combien l'algorithme teste-t-il de combinaisons ? Quel ordre de grandeur de durée met \Python{} pour tester toutes ces possibilités ? (Donne la réponse sous la forme $0.1$ seconde, $1$ seconde, $10$ secondes, $100$ secondes, $1000$ secondes, $10\,000$ secondes\ldots)
  
  Pour le nombre de combinaisons, aide-toi des calculs plus simples suivants :
  \begin{itemize}
    \item Combien y a-t-il de combinaisons avec une seule opération $c_1 \square c_2$ ? Réponse : il y a $6$ choix pour $c_1$ (une plaque parmi les $6$), il y a $4$ opérations possibles, et enfin il y a $5$ choix pour $c_2$ (une plaque parmi les $5$ plaques restantes). Bilan :
    $$6 \times 4 \times 5 = 120 \text{ combinaisons.}$$
     \item Avec deux opérations $(c_1 \square c_2) \circ c_3$, il y a 
     $$6 \times 4 \times 5 \times 4 \times 4 =  1920 \text{ combinaisons.}$$
   \end{itemize}
    
  
  
  
\end{enumerate}

\end{activite}


%%%%%%%%%%%%%%%%%%%%%%%%%%%%%%%%%%%%%%%%%%%%%%%%%%%%%%%%%%%%%%%%
% Activité 3 - Parcours d'arbre
%%%%%%%%%%%%%%%%%%%%%%%%%%%%%%%%%%%%%%%%%%%%%%%%%%%%%%%%%%%%%%%%

\begin{activite}[Parcours d'arbre]
	
\index{arbre}
 \index{algorithme recursif@algorithme récursif}
   
\objectifs{Objectifs : résoudre des problèmes en parcourant des arbres. Cette activité utilise la récursivité.}


 \textbf{Atteindre une somme.}
  
   On se donne une liste de nombres, par exemple $[5,7,11,13]$. À partir de ces nombres on doit obtenir une somme $S$ fixée, par exemple $S=28$. Ici c'est possible par exemple :
   $$5+5+7+11 = 28$$
   \emph{Remarques.} Ici on peut utiliser plusieurs fois le même nombre ; on n'est pas obligé d'utiliser tous les nombres ; il n'y a pas toujours de solution ; il peut aussi y avoir plusieurs solutions.
   
   Pour résoudre ce problème, on teste chacun des entiers possibles, ici $5$, puis $7$, puis $11$, puis $13$. Commençons par tester $5$. Si on choisit $5$ alors la somme qu'il reste à atteindre est $S' = S-5 = 23$. On a donc simplifié le problème. Si on avait choisit $7$ alors la somme qu'il reste à atteindre serait $S'=S-7=21$, etc.
   
   Cela s'écrit sous la forme d'un arbre : les arêtes correspondent à un entier, les sommets à la somme qu'il reste à atteindre. L'arbre se termine quand la somme à atteindre est $0$ (on a gagné) ou est négative (cette branche ne fonctionne pas).
 
Voyons l'exemple de la somme $S=19$ à atteindre avec les entiers $[5,7,11]$.
Sur l'arbre de gauche la somme $S$ avec pour enfants les valeurs $S-5$, $S-7$ et $S-11$. Sur l'arbre de droite on poursuit les calculs.
      
 \begin{minipage}{0.4\textwidth}   
\myfigure{0.85}{\tikzinput{fig_recursivite_12a}}
\end{minipage}
\begin{minipage}{0.59\textwidth}  
\myfigure{0.85}{\tikzinput{fig_recursivite_12b}}
\end{minipage} 

\medskip

Si un sommet a une somme à atteindre négative c'est terminé, le choix testé ne convient pas. Si par contre la somme vaut $0$, c'est que la somme est réalisée. Les entiers à choisir pour réaliser la somme initiale sont ceux lus sur les arêtes.
Sur la figure ci-dessous, on arrive à obtenir un sommet dont la somme est $0$ (dans le carré), les arêtes qui permettent d'atteindre ce sommet portent les poids $5$, puis $7$, puis $7$ (dans les cercles). 
Et effectivement $5+7+7=19$.

\myfigure{0.95}{\tikzinput{fig_recursivite_12c}}

     
    Programme une fonction \ci{atteindre_somme(S,chiffres)} qui renvoie une solution possible.
    Par exemple \ci{atteindre_somme(53,[6,7,15])} renvoie \ci{[6,6,6,6,7,7,15]}
    et effectivement $6 + 6 + 6 + 6 + 7 + 7 + 15 = 53$.
     
    Voilà l'algorithme proposé qui correspond au parcours d'un arbre jusqu'à atteindre la somme $S$.
\begin{algorithme}
\sauteligne 
  
\begin{itemize}
  \item 
  \begin{itemize}
   \item Entête : \ci{atteindre_somme(S,chiffres)}
   \item Entrée : une somme $S$ et une liste d'entiers strictement positifs.
   
   \item Sortie : une combinaison, sous la forme d'une liste, permettant de réaliser la somme ou bien \ci{None} en cas d'échec.
 
   \item Action : fonction récursive.
  \end{itemize}

   \item \emph{Cas terminaux.} Si $S=0$  alors renvoyer la liste vide $[]$ qui va plus tard contenir le parcours gagnant. Si $S<0$, renvoyer \ci{None}.
   
   \item \emph{Cas général.}
   Pour chaque élément $x$ de la liste des chiffres :
    \begin{itemize}
      \item par un appel récursif, on définit \ci{parcours} comme le résultat de 
      \ci{atteindre_somme(S-x,chiffres)},
      \item si \ci{parcours} ne vaut pas \ci{None} alors on lui ajoute $x$ en tête de liste, puis on renvoie \ci{parcours}.
    \end{itemize}
    
    \item Renvoyer \ci{None} (c'est le cas uniquement si rien n'a été renvoyé au cours des instructions précédentes). 
  \end{itemize}
 \end{algorithme}     
  

\end{activite}


%%%%%%%%%%%%%%%%%%%%%%%%%%%%%%%%%%%%%%%%%%%%%%%%%%%%%%%%%%%%%%%%
% Activité 4 - Le compte est bon
%%%%%%%%%%%%%%%%%%%%%%%%%%%%%%%%%%%%%%%%%%%%%%%%%%%%%%%%%%%%%%%%

\begin{activite}[Le compte est bon]
   
\objectifs{Objectifs : programmer la solution du \og{}Compte est bon\fg{} en s'inspirant des deux activités précédentes.}
 
 
Programme une fonction \ci{compte_est_bon(total,chiffres)} qui renvoie une solution possible au problème d'atteindre le total donné avec une liste de nombres donnés.
Par exemple avec les nombres \ci{chiffres = [2, 3, 7, 9, 9, 25]} on peut atteindre \ci{total = 457}.
Avec ces données la fonction \ci{compte_est_bon(total,chiffres)} renvoie :
\mycenterline{\ci{['2+3=5', '5+9=14', '7+25=32', '14*32=448', '9+448=457', '457']}}
Ce qui donne le détail des opérations à effectuer :
$$\big(2+3+9)\times(7+25)+9 = 457$$
Il fallait y penser !


Voici l'algorithme qui permet de résoudre le problème. Il est préférable d'avoir bien compris les deux activités précédentes.
\begin{algorithme}
\sauteligne 
  
\begin{itemize}
  \item 
  \begin{itemize}
   \item Entête : \ci{compte_est_bon(total,chiffres)}
   
   \item Entrée : un total à atteindre et une liste d'entiers strictement positifs.
   
   \item Sortie : une liste d'instructions permettant d'atteindre le total ou bien \ci{None} en cas d'échec.
 
   \item Action : fonction récursive.
  \end{itemize}

   \item \emph{Cas terminaux.} Si le total à atteindre est un élément de la liste \ci{chiffres} alors renvoyer la liste \ci{[total]} qui contient ce total comme seul élément.
Si la liste \ci{chiffres} ne contient aucun élément ou bien un seul élément (qui n'est pas le total) alors renvoyer \ci{None}.
   
   \item \emph{Cas général.}
   Ordonner la liste des chiffres \emph{du plus grand au plus petit}. 
   
   Pour chaque chiffre $c_1$ de la liste, pour chaque chiffre $c_2$ plus loin dans la liste, pour chaque opération parmi  \og{}$+$\fg{}, \og{}$-$\fg{}, \og{}$\times$\fg{} et \og{}$/$\fg{} :
   
    \begin{itemize}
      \item former une nouvelle liste \ci{nouv_chiffres} à partir de \ci{chiffres} en retirant $c_1$ et $c_2$,
      \item noter \ci{calcul} le résultat de l'opération $c_1 + c_2$ ou $c_1\times c_2$,\ldots{} selon l'opération,
      \item on ne continue que si \ci{calcul} est strictement positif et si c'est bien un entier,
      \item on ajoute \ci{calcul} à la liste \ci{nouv_chiffres},
      
      \item  par un appel récursif, on définit \ci{parcours} comme le résultat de 
      \ci{compte_est_bon(total,nouv_chiffres)},     
      \item si \ci{parcours} ne vaut pas \ci{None} alors c'est une liste d'instructions et on lui ajoute en tête de liste une instruction sous la forme d'une chaîne de caractères du type \og{}$c_1 + c_2 = $\,\ci{resultat}\fg{} ou \og{}$c_1 \times c_2 = $\,\ci{resultat}\fg{},\ldots, puis on renvoie \ci{parcours}.
    \end{itemize}
    
    \item Renvoyer \ci{None} (c'est le cas uniquement si rien n'a été renvoyé au cours des instructions précédentes). 
  \end{itemize}
 \end{algorithme} 
 
\bigskip
\emph{Exemple.}
Il faut bien comprendre qu'à chaque appel de la fonction, le total à atteindre reste le même, mais que la liste des chiffres change et diminue.
Voyons un exemple dans lequel il faut atteindre $27$ avec les chiffres $[4,5,6,7]$.
L'appel de la fonction est \ci{compte_est_bon(27,[4,5,6,7])}. Dans le corps de cette fonction,
à un moment, on va tester l'opération $4\times5$, on retire alors $4$ et $5$ de la liste des chiffres (car on n'a plus le droit de les utiliser), mais en échange on rajoute le résultat $20$ ($=4\times5$) que l'on peut utiliser dans la suite. Il s'agit donc maintenant d'atteindre $27$ mais avec les chiffres $[20,6,7]$. L'appel récursif est donc \ci{compte_est_bon(27,[20,6,7])}.
Lors de ce nouvel appel, on va faire l'opération $20+7$ (nous savons déjà que cela va être bon), on remplace les chiffres $20$ et $7$ par le résultat $27$, et on effectue l'appel \ci{compte_est_bon(27,[27,6])}. Comme l'un des chiffres est le total à atteindre, c'est un cas terminal gagnant ! La fonction renvoie l'historique des calculs \og{}$4\times 5 = 20$\fg{}, \og{}$20 + 7 = 27$\fg{}.




\bigskip
\emph{Indications.}
\begin{itemize}
  \item Ordonner la liste des chiffres du plus grand au plus petit est important. Cela permet de faire les opérations dans le bon sens : $7-5$ et pas $5-7$, $6/3$ et pas $3/6$.
  \item Voici comment tester si un nombre est un entier : \ci{isinstance(x,int)}.
\end{itemize}

\bigskip

Voici des exemples à calculer :
\begin{itemize}
  \item avec les chiffres $[2, 4, 5, 6, 8, 10]$ il est possible d'atteindre $850$ et $852$ mais pas $851$,
  \item dresse la liste des totaux compris entre $100$ et $999$ impossibles à atteindre avec ces chiffres,
  \item pour les chiffres $[1, 2, 5, 7, 75, 100]$ vérifie que l'on peut atteindre tous les totaux possibles entre $100$ et $999$ (attention les calculs deviennent longs !),
  \item l'un des pire tirages est $[10, 10, 25, 50, 75, 100]$ pour lequel il y a plus de totaux irréalisables que de totaux possibles.
\end{itemize}

\end{activite}


\end{document}
